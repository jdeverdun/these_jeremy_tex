%%%%%%%%%%%%%%%%%%%%%%%%%%%%%%%%%%%%%%%%%%%%%%%%%%%%%%%%%%%%%%%%%%%%%%%%%%%%%%%%%%%%%%%%%%%%%
%%									Chapitre 1											%
%%%%%%%%%%%%%%%%%%%%%%%%%%%%%%%%%%%%%%%%%%%%%%%%%%%%%%%%%%%%%%%%%%%%%%%%%%%%%%%%%%%%%%%%%%%%%
\chapter{Introduction}
	
\minitoc
	

%%%%%%%%%%%%%%%%%%%%%%%%%%%%%%%%%%%%%%%%%%%%%%%%%%%%%%%%%%%%%%%%%%%%%%%%%%%%%%%%%%%%%%%%%%%%%



% Début du chapitre

\section{Introduction}
		
		La recherche de biomarqueurs originaux issues de l’Imagerie par Résonance Magnétique (IRM) multimodale est un enjeu médical majeur du fait du caractère non-invasif et relativement peu 
couteux de cette technique. L'imagerie par résonnance magnétique nucléaire est en effet un outil d'une grande versatilité qui permet de fournir plusieurs « points de vue » aussi bien concernant 
la structure que la dynamique cérébrale. L’étude de la rhéologie du système vasculaire encéphalique en IRM peut recourir à divers outils non invasifs tels que l’imagerie de perfusion par marquage 
des protons artériels du sang (Arterial Spin Labeling ou ASL -~\cite{Alsop2014}), l’imagerie en contraste de phase ou l’imagerie de susceptibilité (Susceptibility Weighted Imaging, SWI). Ces mesures permettent 
l’estimation de différents paramètres de la rhéologie intracrânienne. On peut alors espérer, dans un contexte pathologique particulier, découvrir dans une des différentes modalités d'imagerie le 
marqueur spécifique qui permettra d'identifier et de suivre la pathologie.

Néanmoins, malgré les succès de cette stratégie, il apparaît naturel de passer aujourd'hui de marqueurs construits à partir d'une seule modalité d'imagerie à des marqueurs multimodaux combinant 
les riches informations issues de plusieurs imageries dynamiques. Cette combinaison peut être recherchée dans une perspective de pure corrélation, à travers des stratégies de type « data mining » (\cite{Lin2012}). 
Cependant, d'un point de vue plus « mécanistique », les différents systèmes circulants cérébraux (liquide cérébro-spinal et système vasculaire) étant fortement couplés, il paraît naturel de tenter une 
modélisation biophysique des écoulements regroupant les différentes informations jusque-là séparées. Sur cette base, les relations entre débit, pression, perfusion, perméabilité, à différentes échelles 
dans le cerveau, pourront être précisées sur des bases physiques bien établies, une telle description cohérente semble être une condition nécessaire à la découverte de nouveaux biomarqueurs robustes et 
bien validés.

Plusieurs groupes ont proposé des modélisations globales des écoulements intracrâniens sous forme de systèmes dynamiques d'équations différentielles non-linéaires (\cite{Sorek1988}). Les écoulements sont décrits par 
un formalisme de compartiments (veines, capillaires, artérioles, artères), dont les caractéristiques géométriques (volume, rayon moyen, longueur moyenne entre deux branchements) sont calculées sur la 
base des données d'imagerie, ou déduites du modèle lui-même. Au sein de ces compartiments, des variables dynamiques couplées (pression, flux) peuvent être calculées, pour des conditions aux limites 
données.

Les échelles spatiales et temporelles de description auxquelles se placent ces modèles en font de très bons candidats pour les applications cliniques et la comparaison détaillée aux données d'imagerie. De plus, leur relative rapidité de calcul - caractéristique des systèmes d'équations différentielles
ordinaires - devrait rendre possible un aller-retour rapide entre données d'imagerie et simulations afin de converger vers les paramètres significatifs que nous recherchons.
Nous nous sommes donc proposé dans ce travail de tenter l'intégration la plus complète possible compte tenu de l'état de l'art des informations concernant la rhéologie intracrânienne au sein 
d'un modèle biophysique réaliste des écoulements. Nous nous sommes fixés comme but de construire ce modèle de façon sujet-spécifique, afin de rendre compte de la variabilité des structures 
impliquées.

Deux conditions doivent être réunies pour rendre possible la réalisation de ce programme. Tout d'abord la maîtrise détaillée des modalités d'imagerie IRM qui peuvent nourrir cette modélisation, 
de leurs possibilités et de leurs limites actuelles. Cette connaissance fixera le degré de détail de la modélisation, aussi bien en termes de résolution (spatiale et temporelle) que de variables 
dynamiques et de caractéristiques structurales accessibles. Elle aboutira à définir un protocole expérimental standard d'imagerie qui regroupe pour un sujet ou un patient donné toutes les informations 
qui seront nécessaires au modèle.

La seconde condition est de choisir la stratégie de modélisation la plus adaptée au type de données disponibles. Une revue de la littérature orientée par cette question est donc importante.
En vue d’aboutir à un modèle sujet-spécifique le plus correct possible il sera essentiel de décrire la structure de la vascularisation cérébrale et du système ventriculaire. Cela fera ainsi l’objet 
d’un chapitre dédié au sein duquel nous chercherons en premier lieu à comprendre comment l’arborescence vasculaire, des artères aux veines en passant par les capillaires, se met en place afin d’assurer 
un apport suffisant en sang aux cellules et quelles en sont les éléments clefs. Nous ferons de même pour le système ventriculaire assurant la circulation du liquide cérébro-spinal. Par la suite, 
nous évoquerons les outils d’imagerie par résonnance magnétique à notre disposition pouvant fournir une information précise sur les structures du système circulatoire pour un sujet donné pour terminer 
par la définition d’un protocole d’acquisition morphologique en accord avec les séquences disponibles dans le service.

Le passage de plusieurs images à une structure 3D complète de l’arbre vasculaire est non trivial (\cite{Luboz2005}). Il requiert l’identification des éléments d’intérêt dans l’image. Chaque imagerie fournira ainsi 
une partie du puzzle permettant d’appréhender l’ensemble de l’architecture. Nous évoquerons cette problématique dans un second chapitre dédié aux algorithmes de reconstructions nous ayant permis tout 
d’abord d’identifier les artères et les veines indépendamment et d’en extraire une représentation 3D sous forme de segments, d’appréhender ensuite les compartiments artériolaires, capillaires, et
veinulaires en vue de faire la jonction entre artères et veines, et enfin de caractériser le système ventriculaire des ventricules latéraux à l’espace sous arachnoïdien.

L’information anatomique fournit la base structurelle au modèle. Pour être complet il doit se nourrir de données de la dynamique des flux intracrâniens, d’une part afin de définir les conditions 
limites et d’autre part afin de valider et interpréter les résultats de la simulation. Le troisième chapitre se focalisera sur l’imagerie dynamique de la circulation intracrânienne où, après quelques 
rappels sur les flux vasculaires et ventriculaires, seront brièvement présentées les principales séquences IRM que nous allons utiliser pour caractériser les flux et le protocole mis en place.
Sur la base de ces données nous pouvons établir un modèle de la circulation intracrânienne. Pour ce faire nous décrirons dans un quatrième chapitre brièvement les modèles à compartiments existants, 
les essais d’intégration des données structurales, et enfin notre modèle, avec les équations qui le composent et les résultats qu’il fournit. Nous tenterons dans ce cadre de les comparer à des travaux 
antérieurs d’autres équipes.

Parmi les séquences utilisées deux d’entre elles sont encore en pleine évolution car relativement récentes. La technique de marquage des protons artériels du sang (Arterial Spin Labeling - ASL) et l’imagerie 
quantitative de susceptibilité magnétique (Quantitative Susceptibility Mapping - QSM). Ces deux techniques sont plus compliquées à mettre en place et ont nécessité une attention toute particulière en vue d’obtenir des résultats de qualité. 
Il semble important de détailler dans des chapitres indépendants notre travail de développement et de validation de ces deux méthodes. Nous commencerons ainsi par l’ASL en décrivant 
les séquences existantes, les méthodes de quantifications de la perfusion cérébrale, les limitations, et une application sur un protocole de recherche. Nous continuerons ensuite par la QSM avec des bases théoriques, l’implémentation que nous avons réalisée et la validation de la mesure.
Nous terminerons par les perspectives d’optimisation du modèle et ses applications.\\

		
\bibliography{jeremythesebib}{}
\bibliographystyle{francaissc}
