%%%%%%%%%%%%%%%%%%%%%%%%%%%%%%%%%%%%%%%%%%%%%%%%%%%%%%%%%%%%%%%%%%%%%%%%%%%%%%%%%%%%%%%%%%%%%
%%									Chapitre 4												%
%%%%%%%%%%%%%%%%%%%%%%%%%%%%%%%%%%%%%%%%%%%%%%%%%%%%%%%%%%%%%%%%%%%%%%%%%%%%%%%%%%%%%%%%%%%%%

\chapter{Chapitre 4}
	\citationChap{
		% Citation
	}{Auteur}
	\minitoc
	\newpage

%%%%%%%%%%%%%%%%%%%%%%%%%%%%%%%%%%%%%%%%%%%%%%%%%%%%%%%%%%%%%%%%%%%%%%%%%%%%%%%%%%%%%%%%%%%%%

On va raconter n'importe quoi à propos des \gls{asb}, juste pour illustrer à quoi ressemblent les différents glossaires. On pourrait tout aussi bien converser sur la pertinence de l'utilisation des \gls{csl} pour caractériser les \glspl{macle} du \gls{rutile}. Et pour craner un peu, je vais citer le merveilleux travail de \citet{depriester2014thermomechanical}. Maintenant que les \gls{asb} et \gls{csl} ont été définies, plus besoin de détailler leurs significations.
		
	\subsection{Tableaux et figures}
	On va ici placer des éléments graphiques (voir tableau~\ref{tab:exemple} et figure~\ref{fig:exemple}), juste pour avoir des entrées dans les listes des figures et des tableaux. On remarquera l'utilisation des sous-figures~\ref{sub:Antibes} et~\ref{sub:SaintJeannet}.
	
	\begin{tableth}
		\caption[Légende courte pour l'exemple de tableau]{Un tableau avec une légende tellement longue que ce serait hideux dans la liste des tableaux}
			\label{tab:exemple}
		\begin{tabular}{c|c}
			Coucou	& Au revoir\\
			\hline
			maman	& papa
		\end{tabular}
	\end{tableth}

	
	
	\subsection{Symboles mathématiques}
	Rien de spécial à propos des math, hormis l'illustration des symboles listés en fin de document, tels \gls{alpha} ou \gls{gamma}, qui peuvent être utilisés indifféremment en mode \emph{in-line} ou dans des équations\footnote{Le lecteur notera que \texttt{hyperref} ajoute un lien cliquable sur chaque entrée des différents glossaires.} :
	\begin{equation}
		\gls{alpha}=\nicefrac{\gls{gamma}}{2}
		\label{eq:alphagamma}
	\end{equation}
Les entrées des glossaires peuvent même être appelés dans des figures (PDF avec surcouche \LaTeX, ou Ti\textit{k}Z).
	

\section{Deuxième paragraphe}		
	\section{Encore un autre exemple}
		\subsection{En plus concis}
			\blindtext
	\section{Conclusion}
		\blindtext
% Début du chapitre


% Début du chapitre
