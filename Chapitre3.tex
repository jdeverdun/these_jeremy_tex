%%%%%%%%%%%%%%%%%%%%%%%%%%%%%%%%%%%%%%%%%%%%%%%%%%%%%%%%%%%%%%%%%%%%%%%%%%%%%%%%%%%%%%%%%%%%%
%%									Chapitre 3												%
%%%%%%%%%%%%%%%%%%%%%%%%%%%%%%%%%%%%%%%%%%%%%%%%%%%%%%%%%%%%%%%%%%%%%%%%%%%%%%%%%%%%%%%%%%%%%

\chapter{Implémentation de la reconstruction morphologique}
	
	\minitoc
	

%%%%%%%%%%%%%%%%%%%%%%%%%%%%%%%%%%%%%%%%%%%%%%%%%%%%%%%%%%%%%%%%%%%%%%%%%%%%%%%%%%%%%%%%%%%%%
La reconstruction de la morphologie des vaisseaux est la première étape dans la construction du modèle. Les différentes modalités d’imagerie exposées au chapitre 2.2 vont assurer l’extraction quantitative à la fois de l’architecture des vaisseaux et des paramètres géométriques associés : longueurs et aire. \\
Les images d’intérêt issues des différentes modalités d’acquisition doivent être rapportées dans le même espace : les acquisitions en contraste de phase et le T1 sont donc co-registrées (voir Appendice 1.2.2) \textcolor{red}{refaire les refs après} avec l’image disposant de la meilleure résolution, l’imagerie par temps de vol. Ce choix permet d’extraire le maximum d’informations. 
%%%
%%%
%%%
\section{Reconstruction du réseau des artères et des veines}
Comme nous l’avons précisé ci-dessus, deux types d’informations doivent être extraites des données structurales : des informations géométriques locales (aire et longueur) mais aussi des informations topologiques sur l’organisation de ce réseau. Le résultat de l’analyse doit donc être de nous fournir un graphe représentant l’arborescence du système. Les deux étapes naturelles sont donc d’abord de segmenter dans nos imageries structurales les parties correspondantes au système circulant puis d’extraire la topologie de ce système. Cet objectif impose des contraintes spécifiques, telles que l’intégrité et la complétude du réseau qui gouverne les choix fait dans les étapes successives.\\
L’un des critères importants à prendre en compte lors de la sélection de la chaîne de traitement est le degré d’implication de l’utilisateur. A quels moments et combien de fois doit il intervenir ? Pour nous, l’idéal serait de réduire au maximum cette intervention, ou du moins de la limiter à des actions simples. 
%%%
%%%
\subsection{Segmentation}
Que ce soit en imagerie par temps de vol ou en contraste de phase, les vaisseaux (artères et veines) apparaissent avec une intensité plus élevée que les autres tissus. L’approche la plus simple est donc de seuiller l’image afin d’en extraire les hypersignaux, et donc les vaisseaux. Cette approche permet de récupérer rapidement une segmentation mais le résultat reste très sensible aux inhomogénéités du signal et au bruit, qui peuvent en diminuer la qualité (Figure~\ref{fig:2_1_segmentation_arteres}).\\
\begin{figure}[!t]
\centering
\includegraphics[width=9cm]{2_1_segmentation_arteres}
\caption{Illustration de la segmentation des artères à partir d’une image de temps de vol. A) Image ToF brute, B) Image seuillée manuellement, C) Image filtrée par le filtre de Frangi, D) Image filtrée seuillée.}
\label{fig:2_1_segmentation_arteres}	
\end{figure}
Un grand nombre d’outils de segmentation existent dédiés aux images IRM. Mais ils recherchent un but spécifique : l’extraction des différents types de tissus présents dans le cerveau, la matière grise, la matière blanche, et le liquide cérébro-spinal (voir Appendice 1.2.4) \textcolor{red}{refaire les refs après}.  Les approches optimisées pour ce genre de segmentation ne peuvent pas être appliqué dans le cadre des vaisseaux. Pour ces derniers des approches dédiées ont été développées (\cite{Lesage2009}). Elles se décomposent toutes en deux étapes, chacune devant être optimisée : le prétraitement des images puis la segmentation elle-même.\\
Les prétraitements consistent à simplifier les informations contenues dans les images d’angiographie IRM. Les traitements génériques disponibles entre lesquels il s’agît de faire un choix ont été développés dans d’autres contextes comme le sous-échantillonnage et la quantification (\cite{Tschirren2005}). Les outils les plus courants restent cependant les filtres améliorant la qualité de l’image. Citons les filtres gradients  (\cite{Koller1995}), les filtres morphologiques (\cite{Wilkinson2001}), ou les filtres à base de matrice Hessienne (\cite{Frangi1998}). \\
A l’issue du prétraitement, les principaux algorithmes de segmentation utilisés dans des contextes vasculaires se regroupent en trois approches principales : la croissance de région, les contours actifs, les lignes médianes (\cite{Lesage2009}). La croissance de région segmente de façon incrémentielle un objet en recrutant les voxels adjacents selon certains critères comme par exemple la variation locale de l’intensité. Elle requiert le plus souvent la définition d’un point initial, qui peut être obtenu via un simple seuillage sur l’image (\cite{Boskamp2004}). Les contours actifs font évoluer une interface sous l’action de différentes forces : des forces externes calculées à partir des intensités de l’image, et des forces internes de type tension de lignes exprimant des contraintes a-priori la géométrie du contour et sa régularité (\cite{McInerney1996}). Les lignes médianes se focalisent sur l’extraction directe du centre des vaisseaux en utilisant des informations de plus haut niveaux telles que la localisation du centre du vaisseau, l’estimation de sa direction et de sa taille caractéristique (\cite{Aylward2002}).\\
Les essais de comparaisons de ces différentes méthodes que nous avons réalisé sur ces différentes images nous ont amené à choisir la combinaison d’un prétraitement Hessien suivit d’un seuillage simple. Le filtre Hessien est en effet extrêmement robuste dans sa capacité à fournir une image très contrastée des vaisseaux, à partir de laquelle un simple seuillage ajusté manuellement (intervention de l’utilisateur) fournit un résultat qui rend superflu l’utilisation de méthodes de segmentation plus sophistiquées. \\
Le filtre utilisant cette méthode est dit filtre de Frangi  (\cite{Frangi1998},~\cite{Manniesing2006}). Ce filtre vise à extraire une quantité mesurant la vraisemblance pour une région d’appartenir à un vaisseau : on parle de  « vesselness » locale de l’image. L’idée est d’identifier dans l’image les structures en forme de tubes (voir Figure~\ref{fig:2_1_segmentation_arteres}).\\
%%%
\begin{figure}[!t]
\centering
\includegraphics[width=15cm]{2_2_tableau_hessienne}
\caption{Schéma possible en 3D selon la valeur de la valeur propre $\lambda_k$ (H=élevé, L=faible, N = bruité, +/- indique le signe de la valeur propre). Les valeurs propres sont ordonnées selon ($|\lambda_1|\leqslant |\lambda_2|\leqslant |\lambda_3|$ (\cite{Frangi1998}).}
\label{fig:2_2_tableau_hessienne}	
\end{figure}
Mathématiquement, on calcule la matrice Hessienne de l’image à partir de ses dérivées secondes (aux différences finies) et on en extrait des valeurs propres. Ces valeurs propres $\lambda_1$, $\lambda_2$ et $\lambda_3$ (à 3D) caractérisent l’anisotropie des intensités de l’image. Les valeurs propres sont classées par ordre croissant du module ($|\lambda_1|\leqslant|\lambda_2|\leqslant|\lambda_3|$). Un voxel appartenant à un vaisseau doit se caractériser par un ellipsoïde Hessien local allongé : $\lambda_1$ faible (idéalement zéro) et $\lambda_2$ et $\lambda_3$ de grandes magnitudes de mêmes signes. Le tableau de la figure~\ref{fig:2_2_tableau_hessienne} résume les relations existantes entre les valeurs propres de la matrice Hessienne et les différentes structures que l’on cherche à détecter. Avec l’imagerie ToF et le contraste de phase, nous recherchons ainsi les structures claires tubulaires. Le résultat de cette étape est une image dont les intensités représentent pour chaque pixel la vraisemblance d’appartenance à un vaisseau (Figure ~\ref{fig:2_1_segmentation_arteres} C). L’image filtrée est de très bonne qualité (Figure~\ref{fig:2_3_filtre_frangi}). L’utilisation de cet outil moins sensible au bruit et offrant des contours plus lisses, facilite donc les étapes postérieures de traitement. Un simple seuillage, ajusté manuellement, suffit à extraire le volume 3D des vaisseaux (Figure ~\ref{fig:2_1_segmentation_arteres} D, Figure 15). Nous n’avons pas besoin de d’utiliser des algorithmes de croissance de région ou autre. L’utilisateur n’a qu’à adapter le seuil, pour extraire les voxels du masque appartenant aux vaisseaux. 
%%%
\begin{figure}[!t]
\centering
\includegraphics[width=13cm]{2_3_filtre_frangi}
\caption{Comparaison entre une projection des intensités maximales sur une image en temps de vol brute à gauche, et l'image filtrée par le filtre de Frangi à droite. L'image filtrée est moins bruitée et met en évidence les vaisseaux d'intérêt.}
\label{fig:2_3_filtre_frangi}	
\end{figure}	

%%%
\begin{figure}[!t]
\centering
\includegraphics[width=13cm]{2_4_segemtation_cont_phase_tof}
\caption{Segmentations 3D obtenues à partir du contraste de phase (veines, bleu) et du ToF (artères, rouge).}
\label{fig:2_4_segemtation_cont_phase_tof}	
\end{figure}	

Comme on l’a vue dans le chapitre précédent, les images par temps de vol fournissent une information artérielle. Le contraste de phase par contre met en évidence à la fois le versant veineux et artériel. Pour segmenter les veines, l’information artérielle doit donc être éliminée des images du contraste de phase. Il est donc indispensable de segmenter en premier lieu l’imagerie par temps de vol, d’identifier les artères, et de les soustraire de l’image en contraste de phase. La même méthodologie sera ensuite appliquée sur cette image.\\
Les volumes 3D obtenus (Figure \ref{fig:2_4_segemtation_cont_phase_tof}) reflètent l’arborescence veineuse et artérielle et fournissent une base solide l’extraction des données géométriques et topologiques. De plus, l’apport de la carte de susceptibilité magnétique se fait ressentir principalement au niveau des petites veines (Figure~\ref{fig:2_5_segmentation_apport_QSM}).
%%%
\begin{figure}[!t]
\centering
\includegraphics[width=13cm]{2_5_segmentation_apport_QSM}
\caption{Illustration de l'apport de l'information de la QSM dans la définition de l'arborescence veineuse. En bleu les veines identifiées en contraste de phase et en rouges celles issues de la QSM.}
\label{fig:2_5_segmentation_apport_QSM}	
\end{figure}	
%%%
%%%
\subsection{Extraction du squelette}	
La segmentation permet d’aboutir à un volume 3D représentant notre arbre vasculaire. Cependant en l’état il ne renseigne pas sur la structure du réseau. Pour identifier les différents segments et points de jonction, et réduire la quantité de données, il est indispensable d’extraire le squelette du masque et de le représenter sous forme de graphe. \\
Le squelette est une représentation très utilisée car il contient et résume les propriétés topologiques de la forme qu’il représente. Il permet  de décrire les objets par un ensemble de lignes fines réduisant sensiblement le volume d’informations à manipuler. Le squelette est généralement défini comme étant l’ensemble des lignes médianes, c’est-à-dire l’ensemble des points équidistants de deux points de la frontière. Notons que sa construction est très sensible à la qualité de la segmentation, en particulier au bruit.  \\
Il existe une grande variété de méthodes permettant de construire des squelettes à partir de formes données. La plus commune est la méthode dite d’extraction de la carte des distances, qui consiste à calculer en chaque point interne à l’objet, la distance à son contour le plus proche. Une fois cela réalisé, les maximums locaux de cette carte sont récupérés et forment le squelette de l’objet (Figure~\ref{fig:2_6_squelettisation}). 
%%%
\begin{figure}[!t]
\centering
\includegraphics[width=13cm]{2_6_squelettisation}
\caption{Création du squelette d'un rectangle par utilisation de la carte des distances. De gauche à droite, le rectangle brut, la carte de distance, et le squelette.}
\label{fig:2_6_squelettisation}	
\end{figure}	

De façon similaire à la méthode des cartes de distances, la méthode des potentiels généralisés génère un champ à l’intérieur de l’objet. Ce champ n’est plus une simple distance mais un potentiel de type Newtonien, déterminé comme la somme des potentiels gravitationnels générés par des masses placées sur les limites de l’objet. Ce type de méthode peut trouver un grand nombre de variantes en changeant la définition du champ scalaire calculé pour les points intérieur du masque et dont on cherche l’extremum. \\
Enfin, la méthode dite d’amincissement itératif consiste à retirer au fur et à mesure les points du contour de la forme par érosion (\cite{Palagyi2002}), tout en préservant ses caractéristiques topologique. Dans cette approche l’axe médian de l’objet est identifié. Les voxels sont itérativement éliminés de la surface du volume si leur suppression n’affecte pas la connectivité de leur voisinage de 3 x 3 x 3, et si ils ont plus d’un voisin dans ce voisinage. Sinon ils sont définis comme des points terminaux. L’érosion doit être réalisée de façon symétrique afin de garantir la position médiane des lignes du squelette.\\
Il existe deux approches majeures dans les méthodes d’amincissement itératif : les filtres et  les arbres de décisions. Les filtres appliquent un élément structurant à l’image et peuvent généralement être étendus à des dimensions supérieures à 3D  (\cite{Jonker2000}). Les méthodes basées sur des arbres de décision sont limitées à des données 2D et 3D, mais sont plus rapides que les filtres morphologiques. \\
Un algorithme de ce type (\cite{Lee1994}) repose sur un processus itératif dans lequel chaque pixel est testé afin de savoir s’il peut être érodé de l’objet. Le pixel peut être supprimé : 
\begin{itemize}
\item S’il est un pixel de surface. Ce test ne considère qu’une des six directions possible en 3D (Nord, Sud, Est, Ouest, Dessus, Dessous) à la fois afin de réaliser l’amincissement de façon symétrique;
\item S’il n’est pas la fin d’une ligne;
\item Si la suppression du point ne change pas la caractéristique d’Euler, par exemple si aucun trou n’est créé lorsque le pixel est retiré (voir table d’Euler de~\cite{Lee1994});
\item Si le point est un point simple, c’est-à-dire qu’il que sa suppression ne change pas le nombre d’objets connectés.
\end{itemize}
Ce processus est réalisé en parallèle pour chaque pixel de l’image, et répété jusqu’à ce qu’il n’y ait plus de changements. Lee et al. (~\cite{Lee1994}) ont démontré dans leurs travaux, que leur solution basée sur un arbre de décision est capable de trouver correctement l’ensemble des points supprimables à chaque itération. Leur algorithme d’érosion est très rapide.\\
Les résultats obtenus via les différentes approches sont visible sur la Figure~\ref{fig:2_7_skeleton_matlab_toolbox}. 
%%%
\begin{figure}[!t]
\centering
\includegraphics[width=13cm]{2_7_skeleton_matlab_toolbox}
\caption{Exemple de génération de squelette à partir d'un objet avec différentes méthodes. A) Objet brut, B) Squelette par amincissement itératif, C) squelette par carte des distances et D) squelette par carte de champ potentiels. Figure générée par utilisation de la toolbox {\tt Volume Skeleton Matlab Toolbox} (Liu, Rutgers University).}
\label{fig:2_7_skeleton_matlab_toolbox}	
\end{figure}	
Comme on le voit ces différentes approches fournissent différents niveaux de détails sur la structure d’intérêt. Il est donc nécessaire d’évaluer au vu de l’objectif quel niveau de détails nous souhaitons atteindre. Dans notre travail, nous devons extraire le squelette de vaisseaux de diamètres extrêmement variables, et parfois très tortueux. Après quelques tests nous avons décidé d’utiliser la méthode d’amincissement itératif décrit par Lee (\cite{Lee1994} voir Figure~\ref{fig:2_7_skeleton_matlab_toolbox} B) via l’implémentation Matlab fournit par Kerschnitzki et associés (\cite{Kerschnitzki2013}).\\
L’utilisation de cet algorithme sur notre segmentation aboutit à une structure reflétant relativement bien notre topologie initiale (Figure~\ref{fig:2_8_structure_extraite}).
%%%
\begin{figure}[!t]
\centering
\includegraphics[width=13cm]{2_8_structure_extraite}
\caption{Application de la méthode d'amincissement itératif à une segmentation artérielle. A gauche la segmentation et à droite le squelette résultant. Le résultat est similaire avec les veines.}
\label{fig:2_8_structure_extraite}	
\end{figure}	
%%%
%%%
\subsection{Création du graphe}
\label{sect:graph_creation}
L’arbre vasculaire contient de nombreux vaisseaux qui se subdivisent ou se rejoignent,  par exemple au niveau du polygone de Willis. Les structures les plus adaptées pour la représentation de la topologie de ces architectures complexes sont les graphes et plus précisément les graphes orientés. En théorie des graphes, un graphe orienté est défini par un ensemble de nœuds (ou sommets) reliés par des arrêtes (ou liens) orientées. Dans le réseau, un nœud pointe vers un autre dans une direction spécifique. Dans notre contexte les liens représentent les segments de vaisseaux et les nœuds les jonctions entre ces vaisseaux. Notons que la description admet des liens « terminaux » appelés branches qui ne se terminent pas eux même par un nœud (Figure~\ref{fig:2_9_branches_noeuds}).\\
%%%
\begin{figure}[!t]
\centering
\includegraphics[width=16cm]{2_9_branches_noeuds}
\caption{Illustration d'un graphe, les cercles représentent les nœuds.}
\label{fig:2_9_branches_noeuds}	
\end{figure}	

Le passage d’un squelette à une structure de graphe peut être réalisé simplement. On définit comme nœud les groupes de voxels disposants de plus de deux voisins chacun. Les groupes de voxels possédants exactement deux voisins appartiennent à des liens ou à des branches (si ils ne se connectent qu’à un nœud \cite{Kerschnitzki2013})]. Cette conversion est standard : nous avons utilisé l’implémentation de Kollmannsberger(\cite{Kerschnitzki2013}). \\
Le passage de la segmentation brute au squelette puis au graphe conduit à l’apparition d’erreurs telles que la création de faux liens de faibles longueurs ou de fausses anastomoses (Figure~\ref{fig:2_10_boucles}). Il convient donc de les détecter et de les retirer. C’est ce que nous avons mis en place.
%%%
\begin{figure}[!t]
\centering
\includegraphics[width=10cm]{2_10_boucles}
\caption{Illustration de l'élimination automatique des erreurs mis en place. De haut en bas l’évolution du graphe, avec élimination des doubles liens directs, des nœuds en série, puis des segments courts.}
\label{fig:2_10_boucles}	
\end{figure}	
On repère tout d’abord les nœuds ayant un double lien direct avec un autre nœud (une distance maximale est utilisée) et on ne conserve que le lien le plus long. On élimine ensuite les nœuds en série de telle sorte à ce que l’on n’ait pas de sous segments pour un même vaisseau. On identifie et supprime les branches les plus courtes (< 2 mm). On réitère ce processus jusqu’à nettoyage complet du graphe. Le résultat est représenté dans la Figure~\ref{fig:2_11_nettoyage_graphe}.\\
%%%
\begin{figure}[!t]
\centering
\includegraphics[width=10cm]{2_11_nettoyage_graphe}
\caption{Exemple de nettoyage automatique d'un graphe artériel. En haut le graphe brut, en bas le graphe nettoyé.}
\label{fig:2_11_nettoyage_graphe}	
\end{figure}	

Cette méthodologie assure une cohérence à la description topologique fournit par le graphe final (Figure~\ref{fig:2_11_nettoyage_graphe}). Notons malgré tout qu’une étape manuelle est indispensable. Au niveau artériel, on demande à l’opérateur 1) de cliquer sur les liens qui lui paraissent complètement incohérents et 2) d’indiquer si la séparation entre les deux artères cérébrales antérieures n’est pas bonne, de manière à dupliquer les segments (un par hémisphère), la segmentation ayant souvent du mal à les séparer.\\
Par ailleurs la construction automatique ultérieure des équations du modèle demandera de re-parcourir ce graphe en disposant d’informations supplémentaires que l’opérateur doit également fournir à ce stade. Au niveau artériel l’utilisateur doit 1) cliquer sur les artères communicantes, s’il y en a, pour les identifier, et 2) indiquer au niveau des principales intersections, les hémisphères (cette information sera propagée dans le reste de l’architecture de proche en proche). Au niveau veineux on demande à l’utilisateur l’identification de l’hémisphère pour les veines latérales.\\
Pour chaque segment il est enfin nécessaire d’extraire les caractéristiques géométriques : volumes, diamètres et longueurs afin de les caractériser et les décrire. Pour ce faire, nous employons une approche simple qui associe à chaque voxel de la segmentation un label qui l’associe au segment le plus proche du squelette en termes e distance euclidienne. On aboutit ainsi à une labélisation en segments du masque des vaisseaux, on récupère ensuite le volume de chaque segment auquel on associe également une longueur identifiée par le nombre de voxels du segment correspondant du squelette. On peut ensuite extraire le rayon en considérant la structure curviligne comme étant de section constante via
\begin{equation}
\label{eq:rayons}
Rayon\,=\,\frac{Volume}{Longueur\,\pi},
\end{equation}
où l’on a négligé les écarts de volumes entre tube curviligne et tube droit. 
%%%
%%%
%%%
\section{Artérioles – capillaires - veinules}
Les informations artérielles et veineuses peuvent être récupérées par IRM. En revanche les données morphologiques sur les artérioles, capillaires et veinules ne sont pas atteignables à la résolution disponible. Ces compartiments représentent une véritable boite noire pour laquelle il est indispensable de faire des hypothèses basées sur la littérature afin de les représenter au mieux dans le modèle.
%%%
%%%
\subsection{Quelle artère avec quelle veine ? }
Dans une première étape nous devons associer une artère et une veine comme cela se produit physiologiquement, l’ensemble constitue un territoire artério-veineux. Il existe une très forte variabilité de ces territoires du côté veineux en particulier (\cite{Reiner2013}). L’information que l’on pourrait souhaiter utiliser serait un atlas des territoires veineux mais cette variabilité rend impossible d’associer ainsi de façon fine une artère à une veine.\\
En l’état de l’imagerie nous avons sélectionné une approche simple consistant à trouver la veine la plus proche de l’extrémité de chaque branche du système artériel, en utilisant une distance euclidienne. Si ensuite une veine n’est toujours pas reliée à une artère, l’artère la plus proche est réciproquement recherchée et associée à cette veine (Figure~\ref{fig:2_12_liens_arteres_veines}).
%%%
\begin{figure}[!t]
\centering
\includegraphics[width=14cm]{2_12_liens_arteres_veines}
\caption{Illustration des liens effectués entre les artères et les veines. Les cercles bleus représentent les extrémités des branches et les lignes bleues fines les liaisons.}
\label{fig:2_12_liens_arteres_veines}	
\end{figure}	
%%%
%%%
\subsection{Estimation des paramètres des tubes}
La distance euclidienne entre artère et veine permet de proposer une longueur totale approchée, qui correspondra à la somme des artérioles, capillaires et veinules.  Nous n’avons aucun moyen d’accéder directement aux informations morphologiques de ces compartiments. Pour limiter le nombre d’hypothèses à réaliser sur le nombre de vaisseaux, il semble plus logique de ne considérer pour chaque couple artère-veine qu’une « super-artériole », qu’un « super-capillaire », et qu’une « super-veinule » regroupant un ensemble N de vaisseaux. Bien que ces compartiments soient difficilement accessible (voir~\ref{sect:microcirculation}), la littérature nous fournit quelques informations sur les rapports de ces différents compartiments via des volumes moyens (\cite{Zagzoule1986}, \cite{Linninger2009}). Parmi celles-ci on retrouve des rapports de longueurs et de volumes. Par ailleurs~\cite{Moody2004} ont estimé la fraction de volume que représentent les artérioles et les capillaires dans la matière grise et blanche par marquage à la phosphatase alcaline. A partir de ces informations et des données issues de la segmentation, il devient possible d’estimer des volumes plausibles pour nos compartiments.\\	
Dans une première étape, nous réalisons une segmentation sur la base de l’imagerie T1 (voir Appendice 1.2.4) \textcolor{red}{revoir cette ref} . Les volumes matière grise et blanche sont ainsi extraits et un masque de ces tissus créé (probabilité matière grise + blanche > 0.9). En associant à chaque voxel de ce masque le label correspondant au couple artère-veine le plus proche (voir fin du paragraphe~\ref{sect:graph_creation}), on reconstruit les territoires. \\
Les volumes de ces territoires permettent grâces aux rapports issues de littérature d’estimer les volumes des compartiments : 
\begin{equation}
\label{eq:volumesac}
V_{c}\,=\,\bigl(V_{s}\, *\,d_{c-a}\bigr)\, *\,\frac{V_{Ref_{c}}}{V_{Ref_{c}}*V_{Ref_{a}}},
\end{equation}
avec $V_c$ et $V_a$  les volumes des capillaires et des artérioles, $V_s$ le volume segmenté, $d_{c-a}$ la densité estimée d'artérioles e tde capillaires dans un volume (selon~\cite{Moody2004}), et les $V_{ref}$ associés étant les volumes moyens des compartiments issus de la littérature.\\
Pour les veinules, nous n’avons pas l’information de densité. Le volume est donc estimé sur la base des volumes capillaires et artérioles précédemment trouvé en résolvant : 
\begin{equation}
\frac{V_{v}}{V_v\,+\,V_a\,+\,V_c}\,=\,\frac{V_{ref_{v}}}{V_{ref_{v}}\,+\,V_{ref_{a}}\,+\,V_{ref_{c}}},
\end{equation}
avec $V_a$ et $V_c$ les solutions de~\ref{eq:volumesac}, et les volmues de référence toujours donnés par la littérature.\\
De même la longueur totale entre artères et veines est subdivisée en trois longueurs (artérioles, capillaires, veinules) sur la base des proportions fournit par la littérature (\cite{Zagzoule1986}). En l’état, faute d’informations structurales complémentaires, nous n’utilisons pas de facteurs géométriques correctifs supplémentaires (de type dimensions fractales), pour tenir compte du caractère ramifié de cette partie de la structure. \\
A partir des volumes et des longueurs de chaque compartiment, les rayons sont déterminés simplement (Équation~\ref{eq:rayons}). \\
Les rayons moyens des différents compartiments peuvent être visualisés (Figure~\ref{fig:2_13_profil_rayons}) et leur différences relatives mis en comparaisons aux données de la littérature (Figure 39) \textcolor{red}{pour comparer avec Zagzoule faut faire une table}. On voit ainsi que des artères aux veines, les proportions des rayons des différents tubes semblent comparable.\\
%%%
\begin{figure}[!t]
\centering
\includegraphics[width=14cm]{2_13_profil_rayons}
\caption{Illustration des rayons moyens dans les différents compartiments du système.}
\label{fig:2_13_profil_rayons}	
\end{figure}	
\textcolor{red}{Petite remarque : il y a dans le manuscrit original une figure 25 qui n'est pas referencee dans le texte!}

\bibliography{jeremythesebib}{}
\bibliographystyle{francaissc}